\documentclass[compress,trans]{beamer} %trans 适合幻灯片,handout适合讲义
\input{rfstyle.tex}

\title{红旗 HA Cluster技术交流}
%\subtitle{红旗\textbf{Linux}技术交流}
%\author{赵卫国}
\institute[北京中科红旗软件技术有限公司] % (optional, but mostly needed)
{北京中科红旗软件技术有限公司}
%\hypersetup{colorlinks, linkcolor=blue, citecolor=blue,
%    urlcolor=blue,
%    plainpages=flase,
%    pdfcreator=tex,
%    bookmarksopen=true,
%    pdfhighlight=/P,
%    pdfauthor={wgzhao <wgzhao@redflag-linux.com>},
%    pdfcreator={XeLaTex},
%    pdftitle={红旗HA Cluster技术交流},
%    pdfkeywords={ha,cluster},
%    pdfstartview=FitH,
%    pdfsubject={红旗HA Cluster 简介},
%    pdfpagemode=UseOutlines,%UseOutlines, %None, FullScreen, UseThumbs
%}


\date{2009.3}


\begin{document}

\maketitle


\section{集群的概念}
\begin{frame}{集群的分类}
\begin{itemize}
\item HA (High Availability) 集群


\item 负载均衡(Load Balance)集群

\item HPC(High Performance Computing)集群

\end{itemize}
\end{frame}

\begin{frame}{HA集群}
\includegraphics[scale=.6]{images/ha/ha-workflow.png} 
\end{frame}

\begin{frame}{负载均衡集群}
\includegraphics[scale=.6]{images/ha/lb.png}
\end{frame}

\begin{frame}{HPC}
\includegraphics[scale=.5]{images/ha/hpc.png}
\end{frame}

\begin{frame}{HA集群的构成}
\begin{itemize}
  \item 单向待机集群
  \item 不同业务双向待机集群
  \item 同一业务双向待机集群
  \item N+N 结构
\end{itemize}
\end{frame}

\begin{frame}{单向待机集群}
\includegraphics[scale=.4]{images/ha/active-standby.png}
\end{frame}

\begin{frame}{不同业务双向待机集群}
\includegraphics[scale=.6]{images/ha/active-active-m.png}
\end{frame}

\begin{frame}{同一业务双向待机集群}
\includegraphics[scale=.6]{images/ha/active-active.png}
\end{frame}

\begin{frame}{N+N结构}
\includegraphics[scale=.6]{images/ha/n+n.png}
\end{frame}

\begin{frame}{HA集群数据共享方式}
\includegraphics[scale=.6]{images/ha/data-share.png}
\end{frame}

\section{红旗HA Cluster特征}
\begin{frame}{红旗HA Cluster特征}
\includegraphics[scale=.6]{images/ha/ha.png}
\end{frame}

\begin{frame}{红旗HA Cluster的产品构成}
\begin{description} 
\item[Cluster Server] 
        集群主体,包含所有服务器的高可用性功能。还包含 WebManager 的服务器一端的功能。
\item[WebManager] 
 	群操作的管理工具。使用 Web 浏览器作为用户接口。实体嵌入在Cluster Server 中 , 通 过 管 理 终
端 上 的 Web 浏 览 器 进 行 操 作 , 据 此 与 Cluster Server 主体区分。
\item[Builder]   
	Cluster 配置信息的创建工具。与 WebManager 相同,使用 Web 浏览器作为用户接口。在使用 Builder 的终端上,Builder
需要与 Cluster Server 分开安 装。
\end{description}
\end{frame}

\begin{frame}{红旗HA Cluster软件构成}
\includegraphics[scale=.6]{images/ha/ha-arch.png}
\end{frame}

\begin{frame}{共享磁盘型的典型配置}
\includegraphics[scale=.5]{images/ha/se-setup.png}
\end{frame}

\begin{frame}{镜像磁盘型的典型配置}
\includegraphics[scale=.5]{images/ha/le-setup.png}
\end{frame}


\begin{frame}[containsverbatim]{字符管理界面}
%\includegraphics[scale=.6]{images/ha/climanager.png}
\fontsize{7pt}{2}
\begin{verbatim}
======================== CLUSTER STATUS ==================
  Cluster : cluster             
  <server>
   *server1 .........: Online           server1
      lanhb1         : Normal           LAN Heartbeat
      lanhb2         : Normal           LAN Heartbeat
      diskhb1        : Normal           DISK Heartbeat
      comhb1         : Normal           COM Heartbeat
    server2 .........: Online           server2
      lanhb1         : Normal           LAN Heartbeat
      lanhb2         : Normal           LAN Heartbeat
      diskhb1        : Normal           DISK Heartbeat
      comhb1         : Normal           COM Heartbeat
  <group>
      oracle  .......: Online           failover group1
      current        : server1
      disk1          : Online           /dev/sdb5
      exec1          : Online           exec resource1
      fip1           : Online           10.0.0.11
     weblogic .......: Online           failover group2
      current        : server2
      disk2          : Online           /dev/sdb6
      exec2          : Online           exec resource2
      fip2           : Online           10.0.0.12
  <monitor>
    diskw1           : Normal           disk monitor1
    diskw2           : Normal           disk monitor2
    ipw1             : Normal           ip monitor1
    pidw1            : Normal           pidw1
    userw            : Normal           usermode monitor
 =============================================================
\end{verbatim}
\end{frame}

\begin{frame}{图形管理界面}
\includegraphics[scale=.5]{images/ha/webmanager.png}
\end{frame}


\section{支持情况}
\begin{frame}{支持的心跳方式}
在服务器之间互相确认生存状态所使用的资源。
现在支持的心跳资源如下所示:
\begin{itemize}
  \item LAN心跳资源 \\  使用Ethernet的通信。
  \item 内核模式LAN心跳资源  \\ 使用Ethernet的通信。
  \item COM心跳资源 \\  使用RS232C(COM)的通信。
  \item 磁盘心跳资源 \\ 使用共享磁盘上的特定分区(磁盘心跳分区)的通信。仅限共享磁盘配置时使用
\end{itemize}
\end{frame}

\begin{frame}{支持的监控对象}
\begin{itemize}
  \item Oracle 9i/10g/11g
  \item DB2	9.x
  \item PostgreSQL >8.0,PowerGres Plus 2.1/5.0
  \item MySQL >5.0
  \item Sybase  >12.5
  \item Samba	>3.0
  \item NFS		3.0/4.0
  \item Apache	>2.0
  \item Sendmail >8.0
  \item Tuxedo	>9.0
  \item OracleAS 10g
  \item WebLogic >9.0
  \item Websphere >6.1
  \item 自定义的应用程序
\end{itemize}
\end{frame}

\section{一些概念}
\begin{frame}{集群对象}

\begin{tabular}{rp{7cm}}
\multicolumn{2}{c}{在红旗HA Cluster中,使用以下结构管理各种资源} \\
\hline \hline

集群对象 & 		配置集群的单位。\\
服务器对象 &		表示实体服务器的对象,属于集群对象。\\
服务器组对象 & 	捆绑服务器的对象,属于集群对象。\\
心跳资源对象 & 	表示实体服务器的NW部分的对象,属于服务器对象。\\
组对象 & 		表示虚拟服务器的对象,属于集群对象。\\
组资源对象 & 	表示拥有虚拟服务器的资源 (NW、磁盘)的对象,属于组对象。\\
监视资源对象 & 	表示监视机构的对象,属于集群对象。\\
\end{tabular}
\end{frame}

\begin{frame}{资源}
在集群中,监视端和被监视端的对象都称为资源,分类进行管理。这样不仅能够明确区分监视/被监视的对象,
     还能够使构建集群或查出故障时的对应更简便。资源分为心跳资源、组资源、监视资源三类。以下简要介绍各类资源。
\end{frame}

\begin{frame}{心跳资源}
在服务器之间互相确认生存状态所使用的资源。
现在支持的心跳资源如下所示:
\begin{itemize}
  \item LAN心跳资源 \\  使用Ethernet的通信。
  \item 内核模式LAN心跳资源  \\ 使用Ethernet的通信。
  \item COM心跳资源 \\  使用RS232C(COM)的通信。
  \item 磁盘心跳资源 \\ 使用共享磁盘上的特定分区(磁盘心跳分区)的通信。仅限共享磁盘配置时使用
\end{itemize}
\end{frame}

\begin{frame}{组资源}
\fontsize{7pt}{2}
   组成失效切换的单位——失效切换组的资源。
   现在支持的组资源如下所示:
\begin{itemize}
  \item 浮动IP资源 (fip) \\ 提供虚拟IP地址。客户端可以像普通IP地址一样访问
  \item EXEC资源 (exec) \\ 提供启动/停止业务(DB、httpd、etc..)的机制
  \item 磁盘资源 (disk) \\ 提供共享磁盘上的指定分区。仅限(共享磁盘)配置时使用
  \item 镜像磁盘资源 (md) \\ 提供镜像磁盘上的指定分区。仅限(镜像磁盘)配置时使用
  \item 共享型镜像磁盘资源 (hd) \\ 提供共享磁盘或磁盘上的指定分区。仅限(共享型镜像磁盘)配置时使用
  \item RAW资源 (raw) \\ 提供共享磁盘上的RAW设备。仅限共享磁盘配置时使用
  \item VxVM磁盘组资源 (vxdg) \\ 提供共享磁盘上的VxVM磁盘组。与VxVM卷资源一起使用。仅限(共享磁盘)配置时使用
  \item VxVM卷资源 (vxvol) \\ 提供共享磁盘上的VxVM卷。与VxVM磁盘组资源一起使用。仅限(共享磁盘)配置时使用
  \item NAS资源 (nas) \\ 连接NAS服务器上的共享资源。(集群服务器并不是作为NAS的服务器端运行的资源。) 
  \item 虚拟IP资源(vip) \\ 提供虚拟IP地址。可以像访问客户端的普通IP地址一样访问虚拟IP地址。用于配置网络地址在不同区间的远程集群。
\end{itemize}
\end{frame}

\begin{frame}[allowframebreaks=.8,allowdisplaybreaks]{监视资源}
是集群系统内进行监视的主体资源。
现在支持的监视资源如下所示:
\begin{itemize}
  \item IP监视资源 (ipw) \\  提供外部IP地址的监视机构。
  \item  磁盘监视资源 (diskw) \\ 提供磁盘的监视机构。也可以用于共享磁盘的监视
  \item 镜像磁盘监视资源 (mdw) \\  提供镜像磁盘的监视机构
  \item 镜像磁盘接口监视资源 (mdnw) \\ 提供镜像磁盘接口的监视机构
  \item 共享型镜像磁盘监视资源 (hdw) \\提供共享型镜像磁盘的监视机构
  \item 共享型镜像磁盘接口监视资源 (hdnw) \\ 提供共享型镜像磁盘接口的监视机构
  \item PID监视资源 (pidw) \\ 提供EXEC资源启动的进程的生存状态监视功能
  \item 用户空间监视资源 (userw) \\ 提供用户空间的停止监视机构
  \item RAW监视资源 (raww) \\ 提供磁盘的监视机构。因为使用RAW设备,read尺寸小,能够减轻系统负担。也可以用
  于监视共享磁盘
  \item NIC Link Up/Down监视资源 (miiw) \\ 提供LAN线缆的链接状态的监视机构
  \item VxVM Deamon监视资源 (vxdw) \\ 提供VxVM的Deamon监视机构。仅限(共享磁盘)结构时使用
  \item VxVM卷监视资源 (vxvolw) \\ 提供VxVM的卷的监视机构。仅限(共享磁盘)结构时使用
  \item Multi-Target监视资源 (mtw) \\ 提供捆绑多个监视资源的状态
  	\item 虚拟IP监视资源(vipw) \\  提供送出虚拟IP资源RIP包的机构
	\item ARP监视资源(arpw) \\    提供送出浮动IP或虚拟IP资源ARP包的机构
	\item DB2监视资源 (db2w) \\ 提供IBM DB2数据库的监视机构
	\item ftp 监视资源 (ftpw) \\ 提供 FTP 服务器的监视机构
	\item http监视资源 (httpw) \\ 提供HTTP服务器的监视机构
	\item imap4 监视资源 (imap4w) \\ 提供 IMAP4 服务器的监视机构
	\item MySQL监视资源 (mysqlw) \\ 提供MySQL数据库的监视机构
	\item nfs监视资源 (nfsw) \\ 提供nfs文件服务器的监视机构
	\item Oracle监视资源 (oraclew) \\ 提供Oracle数据库的监视机构
	\item OracleAS 监视资源(oracleasw) \\提供 Oracle 应用程序服务器的监视机构
	\item pop3 监视资源(pop3w) \\ 提供 POP3 服务器的监视机构
	\item PostgreSQL 监视资源 (psqlw) \\ 提供 PostgreSQL 数据库的监视机构
	\item samba监视资源 (sambaw) \\ 提供samba文件服务器的监视机构
	\item smtp监视资源 (smtpw) \\ 提供SMTP服务器的监视机构
	\item Sybase监视资源 (sybasew) \\ 提供Sybase数据库的监视机构
	\item Tuxedo监视资源 (tuxw) \\ 提供Tuxedo应用程序服务器的监视机构
	\item Websphere监视资源 (wasw) \\ 提供Websphere应用程序服务器的监视机构
	\item Weblogic监视资源 (wlsw) \\ 提供Weblogic应用程序服务器的监视机构
	\item WebOTX 监视资源(otxw) \\ 提供 WebOTX 应用程序服务器的监视机构
\end{itemize}
 \end{frame}

\section{要求}
\begin{frame}{HA Cluster需要的硬件}
\begin{columns}[t]
\column{.5\textwidth}
   RedFlag HA Cluster 在以下架构的服务器上运行。
   \begin{itemize}
     \item        IA32
     \item        x86\_64
     \item        IA64   (不支持Replicator, Replicator DR,Agent,Alert Service)
     \item        ppc64  (不支持Replicator, Replicator DR,Agent,Alert Service)
   \end{itemize}
\column{.5\textwidth}
   RedFlag HA Cluster Server 所需的规格如下所示
   \begin{itemize}
     \item RS-232C板卡 一个(构建3节点以上集群时不需要)
     \item Ethernet板卡两个以上
     \item 共享磁盘
     \item 镜像用磁盘或镜像用剩余分区
     \item CD-ROM驱动器
   \end{itemize}
      
\end{columns}
\end{frame}

\begin{frame}{成功案列}
\begin{itemize}
  \item 中国邮政集团
  \item 中国邮政储蓄银行
  \item 江西财政厅金财工程
  \item 江西省信访局信访一期
  \item 人民银行江西支行同城清算
  \item 湖南省财政厅国库集中支付
  \item 湖南省政府信息中心邮件系统
  \item \ldots \ldots
\end{itemize}
\end{frame}
\end{document}
